%\VignetteIndexEntry{Molecule Identification with CAMERA}
%\VignetteKeywords{CAMERA}\\
%\VignettePackage{CAMERA}\\

\documentclass[a4paper,12pt]{article}
\usepackage{hyperref}
\usepackage[table]{xcolor}
\setlength{\parindent}{0cm}

\newcommand{\Robject}[1]{{\texttt{#1}}}
\newcommand{\Rfunction}[1]{{\texttt{#1}}}
\newcommand{\Rpackage}[1]{{\textit{#1}}}
\newcommand{\Rclass}[1]{{\textit{#1}}}
\newcommand{\Rmethod}[1]{{\textit{#1}}}
\newcommand{\Rfunarg}[1]{{\textit{#1}}}

\newcommand{\denovo}{{\em de-Novo{}}}
\usepackage{varioref}
\labelformat{figure}{\figurename~#1}
\labelformat{table}{\tablename~#1}


\usepackage{Sweave}
\begin{document}
\title{LC-MS Peak Annotation and Identification with \Rpackage{CAMERA}}
\author{Carsten Kuhl, Ralf Tautenhahn and Steffen Neumann}
\maketitle

\section{Introduction}
%{{{

The R-package \Rpackage{CAMERA} is a {\bf C}ollection of {\bf
A}lgorithms for {\bf ME}tabolite p{\bf R}ofile {\bf A}nnotation. Its
primary purpose is the annotation and evaluation of LC-MS data. It includes algorithms 
for annotation of isotope peaks, adducts and fragments in peak lists. 
Additional methods cluster mass signals that originate from a single metabolite, based on
rules for mass differences and peak shape comparison \cite{annobird07}.
To use the strength of already existing programs, CAMERA is designed to interact directly with processed peak data
from the R-package \Rpackage{xcms} \cite{Smith06XCMSProcessingmass}.

Based on this annotation results, the molecular composition can be calculated
if the mass spectrometer has a high-enough accuracy for both the mass
and the isotope pattern intensities in combination with the R-package \Rpackage{Rdisop}

%}}}
\section{Short Background}
%{{{

For soft ionisation methods such as LC/ESI-MS, different kind of ions besides the protonated molecular ion occurs.
These are adducts (e.g.$[M+K]^+$, $[M+Na]^+ $) and fragments (e.g.  $[M-C_3H_9N]^+$,
$[M+H-H_20]^+ $). Depending on the molecule having an intrinsic
charge, $[M]^+$ may be observed as well. In most cases a substance generates a
bulk of different ions. There interpretation is time consuming, especially if
substances co-elute. Therefore deconvolution, which separates the different
substances and discovery of the ion species is necessary. 

The working with CAMERA to solves these problems is demonstrated in the next
chapters.

%}}}

\section{Processing with \Rpackage{CAMERA}}

\subsection{Preprocessing with \Rpackage{xcms}}
\label{preprocess}

CAMERA needs as input an \Rclass{xcmsSet} object that is processed with your
favorite parameters. For an example see below:
\begin{verbatim}
  library(CAMERA)
  #Single sample example
  file <- system.file('mzdata/MM14.mzdata', package = "CAMERA")
  xs <- xcmsSet(file,method="centWave",ppm=30,peakwidth=c(5,10))

  #Multiple sample
  library(faahKO)
  filepath <- system.file("cdf", package = "faahKO")
  xsg <- group(faahko)
  xsg <- fillPeaks(xsg)
\end{verbatim}

After the preprocessing we create an CAMERA object, which is called xsAnnotate
or in short xsa. 
\begin{verbatim}
  library(CAMERA)
  xsa <- xsAnnotate(xs)
\end{verbatim}
Depending on your analysis the upcoming workflows may differ at this point and
we start with the description of the annotation workflow. Afterwards we
demonstrate the wrapper functions and how to interpret the results.

\subsection{Annotation Workflow}

The CAMERA annotation procedure can be split into two parts: We want to
answer the questions which peaks occur from the same molecule and
secondly compute its exact mass and annotate the ion species. 
Therefore CAMERA annotation workflow contains following primary functions:
\begin{enumerate}
 \item peak grouping after retention time (\Rfunction{groupFWHM})
 \item peak group verification with peakshape correlation
(\Rfunction{groupCorr})
\end{enumerate}
Both methods separate peaks into different groups, which we define as
"pseudospectra". Those pseudospectra can consists from one up to 100 ions,
depending on the molecules amount and ionizability.
Afterwards the exposure of the ion species can be performed with:
\begin{enumerate}
 \item annotation of possible isotopes (\Rfunction{findIsotopes})
 \item annotation of adducts and calculating hypothetical masses for the group
(\Rfunction{findAdducts})
\end{enumerate}
This workflow results in a data-frame similar to a \Rpackage{xcms} peak table,
that can be easily stored in a comma separated table
(Excel-readable).

The next section shows some practical examples.

\subsubsection{Working with single sample}
Let's come to the practical work. Here we have a single sample file either in
positive or negative ionization mode.
The \Rclass{xcmsSet} was created as shown in section \ref{preprocess}.

\begin{verbatim}
  # Create an xsAnnotate object
  an   <- xsAnnotate(xs)
  # Group after RT
  anF  <- groupFWHM(an, perfwhm = 0.6)
  # Annotate isotopes
  anI  <- findIsotopes(anF, mzabs = 0.01)
  # Verify grouping
  anIC <- groupCorr(anI, cor_eic_th = 0.75)
  #Annotate adducts
  anFA <- findAdducts(anIC, polarity="positive")
  anFA
 \end{verbatim}

In the above example, we create the pseudospectra according to the peak retentiontime
information. The \Rfunarg{perfwhm} parameter defines the window width, which is
used for matching. Lower it for a smaller windows or set it to a higher value,
if the retention time varies. This step generate 14 pseudospectra.

Afterwards we annotate isotopic peaks, with \Rfunarg{mzabs} as allowed m/z
error. In this example we find 33 isotope peaks, with means the number of
$[M+1],[M+2],\ldots$ ions. $[M+1]$ means the first isotopic peak for the monoisotopic
peak $[M]$. This isotope informations are useful in the next step,
where for every peak in one pseudospectra a pairwise EIC correlation is performed. If
the correlation value between two peaks is higher than the threshold
\Rfunarg{cor\_eic\_th} it will stay in the group, otherwise both are separated.
If the peaks are annotated isotope ions, they will not be divided. This seperates
our 14 pseudospectra into 48.

After the second pseudogroup creating step we now finally do a complete
annotation of adducts. Therefore, the \Rfunarg{polarity} parameter must be set.
The final output of the xsAnnotate object shows all important information.

For further processing we export the results with:
\begin{verbatim}
 peaklist <- getPeaklist(anFA)
 write.csv(peaklist, file='xsannotated.csv')
\end{verbatim}
where file is the ouput filename.

That's so far as for the simple sample approach. Please note that every method
has additional parameters, that are not explicitly mentioned here. Also if your
analysis doesn't need annotations, only a seperation into groups, then simply
stop after groupCorr. The grouping results are stored in the list
\Rfunarg{object@pspectra}, which saves the peakindexes as elements for every group.

\begin{verbatim}
 #xsa is here the result from findAdducts
 peak.idx <- xsa@pspectra[[1]]
 #print the indexes of all peaks from pseudospectrum 1
 cat(peak.idx)
\end{verbatim}

\subsubsection{Working with multiple samples}
In this case we have a multiple samples experiment like replicates of one probe or a
wildtype vs. mutant experiment. As in the previous example, we start with the
already processed xcmsSet-object.
Note: If you want an \Rfunction{diffreport} later on, make sure you
run \Rfunction{fillPeaks} on your xcmsSet before.

As test dataset we use here the faahKO. For more information about the dataset
see http://dx.doi.org/10.1021/bi0480335.
CAMERA contains different approaches with multiple sample
analysis. Here we only show the most common way, for the other strategies see
the xsAnnotate manpage, especially the parameter sample.

\begin{verbatim}
#Create an xsAnnotate object
xsa <- xsAnnotate(xsg)
#Group after RT value of the xcms grouped peak
xsaF <- groupFWHM(xsa, perfwhm=0.6)
#Verify grouping
xsaC <- groupCorr(xsaF)
#Annotate isotopes, could be done before groupCorr
xsaFI <- findIsotopes(xsaC)
#Annotate adducts
xsaFA <- findAdducts(xsaFI, polarity="positive")

#Get final peaktable and store on harddrive
write.csv(getPeaklist(xsaFA),file="result_CAMERA.csv")
\end{verbatim}

Similar to the single sample experiment we group according to retention time, 
group according to EIC correlation and finally annotate isotopes and adducts.
The main difference between our first example is the sample selection.
The sample parameter set either one specific sample (sample=x),
a subset (sample=c(x:y)) or all samples (sample=c(1:nSamples)), 
where nSamples is the number of samples in the xcmsSet,
as selection for the correlation analysis. Since the runtime increases with every sample, CAMERA includes a
automatic selection method (sample=NA), where in groupFWHM a representant sample is choosen for every created pseudospectrum.
The automatic selection is also the default value.

\subsection{Interpretation of the Results}
\label{CAMERAresults}
\begin{table}[ht]
\begin{center}
\begin{tabular}{|c|cc|c|c|l|}\hline
 id & mz & rt & isotopes & adduct & pc \\
\hline
             65 & 176.04 & 280.09 &     &           &  4\\
\rowcolor{blue!20}   76 & 136.05 & 280.43 &[14][M+1]1+  &           & 5\\
\rowcolor{blue!20}   77 & 135.05 & 280.43 &[14][M]1+    &           & 5\\
\rowcolor{blue!20}   74 & 153.06 & 280.43 &         &[M+H]+ 152.05437   & 5\\
\rowcolor{blue!20}   75 & 175.04 & 280.43 &         &[M+Na]+ 152.05437  & 5\\
\rowcolor{blue!20}   73 & 197.02 & 280.76 &         &[M+2Na-H]+ 152.05437   &
5\\
             78 & 377.74 & 286.15 &     &           &  6\\
             79 & 732.5  & 286.49 &     &           &  6\\
\rowcolor{red!20}    83 & 488.32 & 286.82 &     &[M+Na]+ 465.33205  & 7\\
\rowcolor{red!20}    82 & 466.34 & 286.82 &     &[M+H]+ 465.33205   & 7\\
...&&&&&\\
\hline
\end{tabular}
\end{center}
     \caption{{\footnotesize{Example of annotation result for one sample. Columns
with intensity values are omitted. blue-line: annotated group 5, red-line:
annotated group 7}}} \label{tab:int}
\end{table}


\ref{tab:int} shows an cutout example of an annotation result. The columns with the
intensity values are omitted and the rows are ordered by their rt values
 for better readability. The column \textit{pc} shows the result of the peak
correlation based
annotation (independent of the annotations \textit{iso} and
\textit{adduct}). Peaks with the same label are supposed to belong to
the same spectrum. The column \textit{adduct} shows the annotation hypotheses
for the ions species. The value after the brackets
is the estimated molecular mass.

The column \textit{isotopes} contains the annotated isotopes for a monoisotopic
peak. The values in the first square brackets denote the
isotope-group-id(column \textit{id}), the second is the isotope annotation and
the number after the brackets is the charge of the isotope.

\section{Wrapper functions and combination with diffreport}

\subsubsection{The function \Rfunction{annotate}}
\Rfunction{Annotate} is a wrapper function for the annotation workflow. It's similar to \Rfunction{annnoteDiffreport},
but doesn't combine the results with those from the diffreport. With the parameter \Rfunarg{pval\_list} a
handmade preselection of pseudospectra is possible. A
"quick"\ mode is also available, that runs only \Rfunction{groupFWHM} and
\Rfunction{findIsotopes}.
The normal mode runs \Rfunction{groupFWHM}, \Rfunction{findIsotopes},
\Rfunction{group\-Corr} and \Rfunction{findAdducts} in order as mentioned. Every
parameter of these functions also work with \Rfunction{annotate}.
As a small example:
\begin{verbatim}
 #A full annotation run
 xsa <- annotate(xs, perfwhm=0.7, cor_eic_th=0.75,
 ppm=10, polarity="positive")
 #Generate result
 peaklist <- getPeaklist(xsa)
 #Save results
 write.csv(peaklist,file="results.csv")
\end{verbatim}

Similar to the previous example, the grouping is followed by an annotation. But
in contrast we now have additional summaries respectively analysis
functions.
For a comparison and statistical analysis between different sample classes,
\Rpackage{xcms} contains the \Rfunction{diffreport} function. CAMERA can use
this method for a better representation.

\begin{verbatim}
#Run fillPeaks on xcmsSet
xsg.fill <- fillPeaks(xsg)
#Make a diffreport with CAMERA result
diffreport <- annotateDiffreport(xsg.fill)
#Save on harddrive
write.csv(diffreport, file="diffreport.csv")
\end{verbatim}

The \Rfunction{annotateDiffreport} is a wrapper for the \Rpackage{xcms}
\Rfunction{diffreport} function and combines the results from CAMERA. The
resulting table has three different columns, see section \ref{CAMERAresults}.
For a speed up it's possible to preselect pseudospectra or make an automatic
selection based on the diffreport result. As an example that selects only groups with a
fold change higher than 4.

\begin{verbatim}
#Example 1 with creating list of interest from grouped xcmsSet
diffreport <- annotateDiffreport(xsg.fill, quick=TRUE)
#Save results
write.csv(diffreport, file="diffreport.csv")
#Look into the table and select interesting pseudospectra
#e.g. pseudospectra 10,11 and 30
psg_list <- c(10,11,30)
diffreport.annotated <- annotateDiffreport(xsg.fill, psg_list=psg_list,
polarity="positive")

#Example 2 with automatic selection
diffreport.annotated <- annotateDiffreport(xsg.fill, fc_th=4,
polarity="positive")
\end{verbatim}

Both examples generate a data-frame, identical to the normal diffreport result,
but now with three additional result columns from CAMERA. In example 1 we
perform a quick-run, that means we only generate the xsAnnotate object und call
groupFWHM and findIsotopes. From these results we preselect 3 pseudospectra
(10,11,30), taken from the column pc. In the next run, we run annotateDiffreport
again with our list as parameter. An annotation will only be done for these
three groups.
In example 2 we perform an automatic preselection, where the \Rfunarg{fc\_th}
parameter defines a threshold for selecting groups, which contains ions with a
fold change higher than four. For other pseudospectra, no adduct annotation will
be calculated. The fold change value is taken from the diffreport result.
For other parameters see the manpage of \Rfunction{annotateDiffreport}.

\subsection{Visualisation of the Results}
For a graphical presentation of the annotation result CAMERA provides the
function \Rfunction{plotEICs} to visualize the raw data and the function
\Rfunction{plotPsSpectrum} to plot all peaks of a speudospectrum.
The next examples shows the use of both functions.
\begin{Schunk}
\begin{Sinput}
> library(CAMERA)
> file <- system.file("mzdata/MM14.mzdata", package = "CAMERA")
> xs <- xcmsSet(file, method = "centWave", ppm = 30, peakwidth = c(5, 10))
> an <- xsAnnotate(xs)
> an <- groupFWHM(an)
> an <- findAdducts(an, polarity = "positive")
> plotEICs(an, pspec = 2, maxlabel = 5)
\end{Sinput}
\end{Schunk}

\begin{figure}
\begin{center}
\includegraphics{CAMERA-EICPspec1}
\end{center}
\caption{\label{EICpspec1} EICs.}
\end{figure}
\ref{EICpspec1} displays the EICs of all peaks from one pseudospectrum.
With this plot you can manual check if the grouping makes sense. In the other
figure \ref{EICpspec2} you see a typical $m/z$ plot, with labelled, annotated
peaks.
\begin{Schunk}
\begin{Sinput}
> plotPsSpectrum(an, pspec = 2, maxlabel = 5)
\end{Sinput}
\end{Schunk}

\begin{figure}
\begin{center}
\includegraphics{CAMERA-Pspec1}
\end{center}
\caption{\label{EICpspec2} Spectra.}
\end{figure}

%

\section{Function Overview}
This section contains for every CAMERA function a small introduction with an
example. See the manpages for further informations.

\subsection{Function annotate}
\label{fct:annotate}
Wrapper function for the whole annotation workflow. Returns a
xsAnnotate object. It handles also all functions parameters.

\begin{verbatim}
 library(CAMERA)
 file <- system.file('mzdata/MM14.mzdata', package = "CAMERA")
 xs   <- xcmsSet(file, method="centWave", ppm=30, peakwidth=c(5,10))
 xsa  <- annotate(xs)
\end{verbatim}

\subsection{Function annotateDiffreport}
\label{fct:annotateDiffreport}
Wrapper function for the xcms diffreport and the annotate function. Returns a
diffreport with the results from CAMERAs annotation progress. It handles also
all functions parameters.

\begin{verbatim}
 library(CAMERA)
 library(faahKO)
 xs.grp      <- group(faahko)
 xs.fill     <- fillPeaks(xs.grp)
 diffreport  <- annotateDiffreport(xs.fill)
 write.csv(diffreport, file="...")
\end{verbatim}

The combination of the diffreport and the CAMERA result can also be done
without these functions. Therefore diffreports \Rfunarg{sortpval} argument must set to
FALSE. After the combination the sorting after the pvalue can be make up.

\begin{verbatim}
  diffrep      <- diffreport(...)
  xsa.peaklist <- getPeaklist(xsa)
  diffrep.new  <- cbind(diffrep, xsa.peaklist[, c("isotopes", "adduct",
"pcgroup")])
  #Sort after pvalue
  diffrep.new  <- diffrep.new[order(diffrep.new[,"pvalue"]),]
\end{verbatim}

\subsection{Function findAdducts}
\label{sec:findAdducts}
After the peak grouping into pseudospectra with \Rfunction{groupCorr} or
\Rfunction{groupFWHM} the resulting \Rclass{xsAnnotate} can be processed with
\Rfunction{findAdducts}. For every pseudospectra all possible adducts are
calculated based on a provided rule table. As default CAMERA calculate its own
table, which contains every possible combination from the standard ions
H, Na, K, NH4 and CL, depending on your ionization mode.

Additional CAMERA contains four precalculated rule tables: primary\_adducts\_pos,
primary\_adducts\_neg, extended\_adducts\_pos, extended\_adducts\_neg

Those can be applied as shown in the example. For creating your own rule
table, see \ref{sec:ruletable}.

\begin{verbatim}
 file <- system.file('mzdata/MM14.mzdata', package = "CAMERA")
 xs   <- xcmsSet(file, method="centWave", ppm=30, peakwidth=c(5,10))
 an   <- xsAnnotate(xs)
 an   <- groupFWHM(an)
 an   <- findIsotopes(an)  # optional but recommended.
 an   <- findAdducts(an, polarity="positive")
 
 #With provided rule table
 file  <- system.file('rules/primary_adducts_pos.csv', package = "CAMERA")
 rules <- read.csv(file)
 an    <- findAdducts(an, polarity="positive", rules=rules)
\end{verbatim}

\subsection{Function findIsotopes}
\label{sec:findIsotopes}
For a provided xsAnnotate, CAMERA can identify isotope peaks within every
pseudospectra. The function \Rfunction{findIsotopes} takes as parameter
\Rfunarg{maxcharge} and \Rfunarg{maxiso} that controls the maximum number of
the expected isotopes within one cluster and their charges.

\begin{verbatim}
 library(CAMERA)
 file <- system.file('mzdata/MM14.mzdata', package = "CAMERA")
 xs   <- xcmsSet(file, method="centWave", ppm=30, peakwidth=c(5,10))
 an   <- xsAnnotate(xs)
 an   <- groupFWHM(an)
 an   <- findIsotopes(an)
\end{verbatim}

\subsection{Function findNeutralLoss}
\label{sec:findNeutralLoss}
A common strategy to identify interesting compounds is the screening after
specific neutral losses. CAMERA adopts this strategy and provides with
\Rfunction{findNeutralLoss} and \Rfunction{findNeutralLossSpecs} an interface
for scanning every pseudospectrum for neutral losses. The difference between
both methods is in the results. \Rfunction{findNeutralLossSpecs} returns an
artificial xcmsSet containing the peaks, which have the neutral loss. 
\begin{verbatim}
  library(CAMERA)
  file <- system.file('mzdata/MM14.mzdata', package = "CAMERA")
  xs   <- xcmsSet(file, method="centWave", ppm=30, peakwidth=c(5,10))
  an   <- xsAnnotate(xs)
  an   <- groupFWHM(an)
  xs.pseudo <- findNeutralLoss(an,mzdiff=18.01,mzabs=0.01) 
  #Searches for Peaks with water loss
  xs.pseudo@peaks #show Hits

\end{verbatim}

\subsection{Function findNeutralLossSpecs}
\label{sec:findNeutralLossSpecs}
\Rfunction{findNeutralLossSpecs} returns 
a boolean vector for every pseudospectrum, where a hit is marked with TRUE.
\begin{verbatim}
  library(CAMERA)
  file <- system.file('mzdata/MM14.mzdata', package = "CAMERA")
  xs   <- xcmsSet(file, method="centWave", ppm=30, peakwidth=c(5,10))
  an   <- xsAnnotate(xs)
  an   <- groupFWHM(an)
  hits <- findNeutralLossSpecs(an,mzdiff=18.01,mzabs=0.01) 
  #Searches for pseudspecta with water loss
\end{verbatim}

\subsection{Function getIsotopeCluster}
\label{sec:getIsotopeCluster}
This method extracts the isotope annotation from a xsAnnotate object. The order
of the resulting list correspond to those from the whole peaklist.

\begin{verbatim}
  library(CAMERA)
  file <- system.file('mzdata/MM14.mzdata', package = "CAMERA")
  xs   <- xcmsSet(file, method="centWave", ppm=30, peakwidth=c(5,10))
  an   <- xsAnnotate(xs)
  an   <- groupFWHM(an)
  an   <- findIsotopes(an) 
  isolist <- getIsotopeCluster(an)
  isolist[[10]] #get IsotopeCluster 10
\end{verbatim}

See the manpage for an example interaction with \Rpackage{Rdisop} to
calculate the molecular composition.

\subsection{Function getPeaklist}
\label{sec:getPeaklist}
This function returns a peaklist containing all information from an xsAnnotate
object. This peaklist can be directly saved as an csv file.

\begin{verbatim}
  library(CAMERA)
  file <- system.file('mzdata/MM14.mzdata', package = "CAMERA")
  xs   <- xcmsSet(file, method="centWave", ppm=30, peakwidth=c(5,10))
  xsa   <- xsAnnotate(xs)
  xsa   <- groupFWHM(xsa)
  xsa   <- findIsotopes(xsa)
  xsa   <- findAdducts(xsa, polarity="positive")
  peaklist <- getPeaklist(xsa)
  write.csv(peaklist,file="...")
\end{verbatim}

\subsection{Function getpspectra}
\label{sec:getpspectra}
This function returns for a provided pseudospectrum index its peaktable and
CAMERA's annotation information. This peaktable can be directly saved as an
csv file.
Note: The indixes for the isotopes, are those from the whole peaklist. See
\ref{sec:getPeaklist}.
\begin{verbatim}
  library(CAMERA)
  file <- system.file('mzdata/MM14.mzdata', package = "CAMERA")
  xs   <- xcmsSet(file, method="centWave", ppm=30, peakwidth=c(5,10))
  xsa   <- xsAnnotate(xs)
  xsa   <- groupFWHM(xsa)
  psp.peaks <- getpspectra(xsa, 1)
  psp.peaks
\end{verbatim}

\subsection{Function groupCorr}
\label{sec:groupCorr}
This function calculates the pearson correlation coefficient based on the peak
shapes of every peak in the pseudospectrum to separate co-eluted substances.
It's recommended to use groupFWHM before, otherwise the runtime is very long!!

\begin{verbatim}
  library(CAMERA)
  file <- system.file('mzdata/MM14.mzdata', package = "CAMERA")
  xs   <- xcmsSet(file, method="centWave", ppm=30, peakwidth=c(5,10))
  xsa   <- xsAnnotate(xs)
  xsa   <- groupFWHM(xsa)
  xsa   <- groupCorr(xsa)
\end{verbatim}

\subsection{Function groupFWHM}
\label{sec:groupFWHM}
For grouping peaks into pseudospectra, this function uses the retention time
information. Every peaks that falls into a defined window are considered as one
group. The window is defined as a percentage of the peak FWHM around the
RT\_med value.

\begin{verbatim}
  library(CAMERA)
  file <- system.file('mzdata/MM14.mzdata', package = "CAMERA")
  xs   <- xcmsSet(file, method="centWave", ppm=30, peakwidth=c(5,10))
  xsa   <- xsAnnotate(xsa)
  xsa   <- groupFWHM(xsa)
\end{verbatim}

\subsection{Function plotEICs}
\label{sec:plotEICs}
  This method returns a batch plot including the extracted ion chromatograms to
  the current graphics device for a provided pseudospectrum.

\begin{verbatim}
  #Plot all peak EICs of pseudospectrum 1
  plotEICs(xsa,1)
\end{verbatim}

\subsection{Function plotPsSpectrum}
\label{sec:plotPsSpectrum}
  This method plots the spectrum of a pseudospectrum, with labelling the most
  intense peaks.

\begin{verbatim}
  #Plot the spectrum of pseudospectrum 1 and highlight the
  #annotation and mz labels of the 10 strongest peaks
  plotPsSpectrum(xsa,1,maxlabel=10)
\end{verbatim}

\section{Create own ruletable}
\label{sec:ruletable}
As starting point for creating a specific rule table CAMERA provides four rule tables
with primary adducts for positive and negative mode. The saving path can be found 
in R, see below.
\begin{verbatim}
 file1  <- system.file('rules/primary_adducts_pos.csv', package = "CAMERA")
 file2  <- system.file('rules/primary_adducts_neg.csv', package = "CAMERA")
 file3  <- system.file('rules/extended_adducts_pos.csv', package = "CAMERA")
 file4  <- system.file('rules/extended_adducts_neg.csv', package = "CAMERA")
\end{verbatim}

Those files can be edited in every csv editor (e.g. Excel). The rule table has 7 columns.
name: adduct name
nmol: Number of molecules (xM) included in the molecule
charge: charge of the molecule
massdiff: mass difference without calculation of charge and nmol (CAMERA will do this automatically)
oidscore: adduct index. Molecules with the same kations (anions) configuration and different nmol values have the same oidscore. For example [M+H] and [2M+H]
quasi: Every annotation with belong to one molecule is called annotation group, for example [M+H] and [M+Na] where M means the same molecule. A annotation group
must include at least one ion with quasi set to 1 for this adduct. If a annotation group only includes optional adducts ( rule set to 0) then this group is excluded.
To disable this reduction, set all rules to 1 or 0.
ips: Rule score. If one peak can be explained with more than one annotation group, then only this group survice, with has the higher score (sum of all annotation).
This decreases the number of false positive creatly, but the optional settings can differ in each machine.

After creation of your own rule set, use it as parameter rules in findAdducts see \ref{sec:findAdducts}.

\begin{thebibliography}{t1}
\bibitem{annobird07} Ralf Tautenhahn, Christoph B\"ottcher, Steffen
  Neumann : Annotation of LC/ESI--MS Mass Signals, BIRD 2007 Proc. of
  BIRD 2007 -- 1st International Conference on Bioinformatics Research
  and Development, 2007.
  \url{http://www.springerlink.com/content/473l404001787974/}
  and \url{http://msbi.ipb-halle.de/~rtautenh/bird07.pdf}

\bibitem{Smith06XCMSProcessingmass}
Smith,~C.; Want,~E.; O'Maille,~G.; Abagyan,~R.; Siuzdak,~G. \emph{Anal Chem}
  \textbf{2006}, \emph{78}, 779--787\relax


\end{thebibliography}

\end{document}
